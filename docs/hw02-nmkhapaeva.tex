% hsenlptemplate.tex (Version 2.0)
% ===============================================================================
% NATIONAL RESEARCH UNIVERSITY HIGHER SCHOOL OF ECONOMICS NLP Course HW LaTeX template.
% 2016; Natalia Khapaeva, Higher School of Economics
% Licence type: Free as defined in the GNU General Public Licence: http://www.gnu.org/licenses/gpl.html

\documentclass[a4paper,12pt,fleqn]{article}
\usepackage{amsmath}
\usepackage{fancyhdr}
\usepackage{paralist}

% Insert your course information here %%%%%%%%%%%%%%%%%%%%%%%%%%%%%%%%%%

\newcommand{\institution}{National Research University Higher School of Economics}
\newcommand{\titlehd}{RESEARCH SEMINAR}
\newcommand{\examtype}{First Semester, Master program "Bid Data System"”}
\newcommand{\examdate}{November 2016}
\newcommand{\examcode}{Homework 1}
\newcommand{\readtime}{Natalia Khapaeva}
\newcommand{\writetime}{Vladimir A. Fomichov,\\Ph.D., Doctor of Technical Sciences}
\newcommand{\lastwords}{End of Homework}

%%%%%%%%%%%%%%%%%%%%%%%%%%%%%%%%%%%%%%%%%%%%%%%%%%%%

%\setcounter{MaxMatrixCols}{10}
\newtheorem{theorem}{Theorem}
\newtheorem{acknowledgement}[theorem]{Acknowledgement}
\newtheorem{algorithm}[theorem]{Algorithm}
\newtheorem{axiom}[theorem]{Axiom}
\newtheorem{case}[theorem]{Case}
\newtheorem{claim}[theorem]{Claim}
\newtheorem{conclusion}[theorem]{Conclusion}
\newtheorem{condition}[theorem]{Condition}
\newtheorem{conjecture}[theorem]{Conjecture}
\newtheorem{corollary}[theorem]{Corollary}
\newtheorem{criterion}[theorem]{Criterion}
\newtheorem{definition}[theorem]{Definition}
\newtheorem{example}[theorem]{Example}
\newtheorem{exercise}[theorem]{Exercise}
\newtheorem{lemma}[theorem]{Lemma}
\newtheorem{notation}[theorem]{Notation}
\newtheorem{problem}[theorem]{Problem}
\newtheorem{proposition}[theorem]{Proposition}
\newtheorem{remark}[theorem]{Remark}
\newtheorem{solution}[theorem]{Solution}
\newtheorem{summary}[theorem]{Summary}
\newenvironment{proof}[1][Proof]{\noindent\textbf{#1.} }{\ \rule{0.5em}{0.5em}}

% ANU Exams Office mandated margins and footer style
\setlength\parindent{0pt}
\setlength{\topmargin}{0cm}
\setlength{\textheight}{9.25in}
\setlength{\oddsidemargin}{0.0in}
\setlength{\evensidemargin}{0.0in}
\setlength{\textwidth}{16cm}
\pagestyle{fancy}
\lhead{}
\chead{}
\rhead{}
\lfoot{}
\cfoot{\footnotesize{Page \thepage \ of \pageref{finalpage} -- \titlehd \ (\examcode)}}
\rfoot{}

% DEPRECATED: ANU Exams Office mandated margins and footer style
%\setlength{\topmargin}{0cm}
%\setlength{\textheight}{9.25in}
%\setlength{\oddsidemargin}{0.0in}
%\setlength{\evensidemargin}{0.0in}
%\setlength{\textwidth}{16cm}
%\pagestyle{fancy}
%\lhead{} %left of the header
%\chead{} %center of the header
%\rhead{} %right of the header
%\lfoot{} %left of the footer
%\cfoot{} %center of the footer
%\rfoot{Page \ \thepage \ of \ \pageref{finalpage} \\
%       \texttt{\examcode}} %Print the page number in the right footer

\renewcommand{\headrulewidth}{0pt} %Do not print a rule below the header
\renewcommand{\footrulewidth}{0pt}


\begin{document}

% Title page

\begin{center}
%\vspace{5cm}
\large\textbf{\institution}
\end{center}
\vspace{1cm}

\begin{center}
\textit{ \examtype -- \examdate}
\end{center}
\vspace{1cm}

\begin{center}
\large\textbf{\titlehd}
\end{center}

\begin{center}
\large\textbf{\examcode}
\end{center}
\vspace{4cm}

\begin{center}
\textit{Student: \readtime}
\end{center}
\begin{center}
\textit{Supervisor:  \writetime}
\end{center}
\begin{center}
\vfill
MOSCOW\\
2016\\
\end{center}

% End title page



\newpage
\begin{quote}
\textit{\textbf{Deadline:} 28.11.2016\\
Invent a discourse containing 2 or 3 sentences (like Disc1).\\
It should pertain to a real field of professional activity. The invented discourse is to describe 2 or more events. There are casual and time relations between these events.\\
\textbf{Objective 1:} Define a logical basis LogBS fro building a semantic representation of the considered text.\\
\textbf{Objective 2:} construct a formula Semrepr from (Formulas(LogBS)) being a possible SR of your discourse.}
\end{quote}
\\
\textbf{Disc:} In the 1970s Donald Chamberlin developed SEQUEL language. It was used for creation the PL/pgSQL language in 1998.\\

\vspace{1cm}

\textbf{Entities:}
\begin{compactenum}
\item \textit{e1} - the mark of the event "release of SEQUEL language"
\item \textit{e2} - the mark of the event "release of PL/pgSQL language"
\item \textit{x1} - the mark of the person "Donald Chamberlin"
\item \textit{x2} - the mark of SQL language creation
\item \textit{x3} - the mark of PL/pgSQL language creation
\item \textit{t1} - the moment of e1
\item \textit{t2} - the moment of e2
\end{compactenum}

\vspace{1cm}

\textbf{Set of constants:}
\begin{compactenum}
\item \textit{Agent} - a designation of the semantic role realized in the fragment “Donald Chamberlin developed”
\item \textit{New-idea} - a designation of the semantic role realized in the fragment “developed SEQUEL language”
\item \textit{Time} - a designation of the semantic role realized in the fragments “In the 1970s ... developed”, “was used ... in 1998”
\end{compactenum}

\vspace{1cm}

\textbf{Predicates}
\begin{compactenum}
\item person
\item development
\item creation
\item 1970s/years
\item "SEQUEL language"
\item "PL/pgSQL language"
\item Donald
\item Chamberlin
\item 1998/year
\item language
\end{compactenum}

\vspace{1cm}

\textbf{Binary predicates}
\begin{compactenum}
\item Sem-descr (e1, development)
\item Sem-descr (e2, creation)
\item Sem-descr (x1, person)
\item Sem-descr (x2, development)
\item Sem-descr (x3, "PL/pgSQL language")
\item Name (x1, "Donald")
\item Surname (x1, "Chamberlin")
\item Name1(x2, "SEQUEL language development")
\item Name1 (x3, "PL/pgSQL language creation")
\end{compactenum}

\vspace{1cm}

So, a possible semantic representation (SR) of the first sentence can be constructed as the following formula Sem1: \\
\begin{verbatim}
(((Sem-descr(e1, development)
    AND Time(e1, t1))
    AND (Agent (e1, x1)
    AND (Sem-descr(x1, person)))
AND (Name(x1, "Donald")
AND Surname (x1, "Chamberlin"))
AND New-idea(e1, x2)
AND (Sem-descr (x2, development)
    AND Name1("SEQUEL language development")
    AND Part-time(t1, 1970s)))
\end{verbatim}

Similarly, a possible semantic representation (SR) of the second sentence Sem2:\\
\begin{verbatim}
(((Sem-descr (e2, creation)
        AND Time (e2, t2))
    AND ((Sem-descr(x3, "PL/pgSQL language")
        AND Part-time (t2, 1998))
\end{verbatim}

And possible SR of the discourse Disc1 can be constructed as the following formula Semrepr1: \\
 Semrepr1 = \exists e1 \exists t1 \exists e2 \exists{t2} \exists{x1} \exists{x2} \exists{x3}
 \exists{x4} (Sem1 AND Sem2)

\begin{center}
\vspace{3cm}
--------- \textit{\lastwords} ---------
\end{center}
\\

\label{finalpage}

\end{document}
